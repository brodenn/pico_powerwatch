\documentclass[a4paper,12pt]{article}
\usepackage[T1]{fontenc}
\usepackage[utf8]{inputenc}
\usepackage{geometry}
\usepackage{listings}
\usepackage{xcolor}
\usepackage{hyperref}
\geometry{margin=25mm}

% Code formatting
\lstset{
  basicstyle=\ttfamily\small,
  breaklines=true,
  frame=single,
  backgroundcolor=\color{gray!10},
  keywordstyle=\color{blue},
  commentstyle=\color{green!40!black},
  stringstyle=\color{red!50!brown}
}

\title{Embedded Project Report -- Pico 2 W Sensor Dashboard}
\author{Your Name}
\date{Course: Hårdvarunära programmering 2 -- del 2 \\ Submission: 2025-10-10}

\begin{document}
\maketitle

\section*{Abstract}
This project demonstrates a complete embedded software solution on the Raspberry Pi Pico 2 W using MicroPython.  
It integrates sensor reading, display handling, user input via a rotary encoder, and robust MQTT communication.  
The focus of this report is on the code structure, scheduling strategy, and communication logic.

\section{Introduction}
The task was to build a system that measures a physical quantity, provides a local menu system, and publishes data to an MQTT broker:contentReference[oaicite:0]{index=0}.  
The solution was implemented entirely in MicroPython, with emphasis on asynchronous programming and robust interrupt handling.

\section{System Overview}
The hardware consists of:
\begin{itemize}
    \item Raspberry Pi Pico 2 W
    \item SSD1309 OLED (SPI)
    \item INA219 sensor (I\textsuperscript{2}C) for voltage/current
    \item EC11 rotary encoder with push button:contentReference[oaicite:1]{index=1}
\end{itemize}
The main focus, however, is the software implementation.

\section{Code Structure}
The software is divided into logical modules:

\subsection{Measurement Task}
Samples voltage and current every 100 ms and calculates derived values such as power, Wh, and mAh.

\begin{lstlisting}[language=Python, caption=Measurement task]
async def task_measure(ms=100):
    global v,i,p,tprev
    while True:
        now=time.ticks_ms()
        dt=max(0.001, time.ticks_diff(now,tprev)/1000.0)
        tprev=now
        v = ina.voltage()
        i = ina.current()
        p = v*i
        trip_update(v,i,dt)
        await aio.sleep_ms(ms)
\end{lstlisting}

\subsection{OLED User Interface}
Three modes: Live values, Trip statistics, and Menu navigation. The interface is updated asynchronously.

\subsection{Rotary Encoder Handling}
Implemented in raw mode without debounce. A single scheduled drain avoids 
``schedule queue full'' errors while ensuring every edge is processed.

\begin{lstlisting}[language=Python, caption=RotaryRaw ISR scheduling]
def _try_schedule(self):
    if not self._scheduled:
        self._scheduled = True
        try:
            micropython.schedule(self._drain, 0)
        except RuntimeError:
            self._scheduled = False
\end{lstlisting}

\subsection{MQTT Client}
Implements auto-reconnect, last will, periodic ping, and publishes sensor data every 500 ms.  
Topics include voltage, current, power, Wh, mAh, elapsed time, and UI page state.

\begin{lstlisting}[language=Python, caption=MQTT publish logic]
if time.ticks_diff(now, last_pub) >= PUBLISH_MS:
    client.publish(TOPIC_V, b"%.3f" % v)
    client.publish(TOPIC_I, b"%.4f" % i)
    client.publish(TOPIC_P, b"%.3f" % (v*i))
    client.publish(TOPIC_WH, b"%.5f" % trip_wh())
    client.publish(TOPIC_MAH, b"%.1f" % trip_mah())
\end{lstlisting}

\section{Results}
\begin{itemize}
    \item Measurements were updated at 0.5 s intervals on the broker.
    \item OLED menus and trip reset worked reliably.
    \item Rotary encoder events were stable despite the lack of debounce.
    \item The system recovered from Wi-Fi or MQTT disconnects automatically.
\end{itemize}

\section{Conclusion}
The project meets all requirements: real sensor measurement, robust MQTT publication, 
and a working local menu system.  
The main challenge was reliable encoder handling, which was solved using a raw interrupt approach with 
a scheduled drain mechanism.  
The code structure makes it easy to extend with additional sensors or menu options.

\end{document}
